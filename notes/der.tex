\documentclass[11pt, oneside]{article}   	% use "amsart" instead of "article" for AMSLaTeX format
\usepackage{geometry}                		% See geometry.pdf to learn the layout options. There are lots.
\geometry{letterpaper}                   		% ... or a4paper or a5paper or ... 
%\geometry{landscape}                		% Activate for rotated page geometry
%\usepackage[parfill]{parskip}    		% Activate to begin paragraphs with an empty line rather than an indent
\usepackage{graphicx}				% Use pdf, png, jpg, or eps§ with pdflatex; use eps in DVI mode
								% TeX will automatically convert eps --> pdf in pdflatex		
\usepackage{amssymb}
\usepackage{amsmath}
\usepackage{parskip}

\title{Bayesian Opt Acquisition Function}

\begin{document}
\maketitle

\paragraph{Data}
The relevant data when running bayesian optimization (BO) for minimization of an unknown function using a gaussian process (GP) prior is:
\begin{itemize}
\item $X = (x_1, \dots, x_n)^T$ are the observations. Each $x_j$ is a $D$-dimensional row vector.
\item $Y = (y_1, \dots, y_n)$ are the values sampled at the observation points. Each $y_i$ is a scalar thus $Y$ is $(D, 1)$ vector. 
\item $y_{best}$ is the current lowest value in $Y$.
\item $x^*$ test point. $x^*$ is itself a $D$-dimensional row vector
\end{itemize}

\paragraph{Kernel}
The Kernel is the core of the GP because it gives the correlation between data points. We usually assume an exponential kernel:
$$ K(x_i, x_j) = \sigma^2 \exp^{-\frac{||x_i - x_j||_2^2}{2l^2}} $$
where $\sigma$ and $l$ are hyperparameters that decide the intensity and the length of the correlation between variables $x_i$ and $|| \cdot ||_2$ is the 2-norm so that any $D$ dimensional data is given one number from the Kernel.

In particular we have that:
\begin{itemize}
\item $K(X, x^*)$ kernel: $(n \times 1)$ matrix
\item $\frac{\partial K(X, x^*)}{\partial x^*}$ kernel derivative: $(n \times D)$ matrix given by
\begin{equation}
\frac{\partial K(X, x^*)}{\partial x^*} = \frac{(X - x^*)}{\ell^2} * K(X, x^*)
\end{equation}
where $(X - x^*) = (x_1 - x^*, \dots, x_n - x^*)^T$ and $*$ is the element wise product in python.
\end{itemize}

\paragraph{Acquisition Function}
The acq function is a scalar function evaluable at a test point $x^*$ that tells us how likely that point is to be an extremal point (minimum or mazimum depending on the optimization problem). So to find its form we create a \textit{utility function} that tells how 'good' is a test point $x^*$ in our optimization routine (how much lower it is from the current lowest value), this is intuitively given by: $UF = \min (f(x^*) - y_{best}, 0)$. We conveniently reparametrize it:
$$UF = \min \bigg(\frac{f(x^*) - \mu(x^*)}{\sigma(x^*)} - \frac{y_{best} - \mu(x^*)}{\sigma(x^*)} , 0\bigg)\sigma(x^*)$$
A typical acq function is the Expected Improvement (EI) which is the expectation value of $UF$, namely: $EI = \mathbb{E}[UF]$. Doing the calculation we obtain:
\begin{itemize}
\item acquisition function
$$ a(z, x^*) = -\sigma(x^*) \ \left[ \phi(z)  + z \Phi(z)\right]$$
where $$z=\frac{y_\text{best} - \mu(x^*)}{\sigma(x^*)}$$
with $$ \mu(x^*) = K(X, x^*) K(X, X)^{-1} Y $$, $$ \sigma (x^*) = K(x^*, x^*) - K(x^*, X)K(X, X)^{-1}K(X, x^*) $$
and $\phi(z)$ is a Gaussian centered in 0 and variance 1 while $\Phi(z)$ is its cumulative function.
\item acquisition function derivative
$$\frac{d{a(z, x^*)}}{d x^*} = -\Phi(z) \frac{d z}{d x^*} - \left[\phi(z) +z \Phi(x)\right] \frac{d{\sigma}}{d x^*}$$
with
$$ \frac{d z}{d x^*} = -\frac{1}{\sigma} \frac{d \mu}{d x^*} - \frac{y_\text{best} - \mu}{\sigma^2} \frac{d{\sigma}}{d x^*}$$
\end{itemize}
\end{document}  
